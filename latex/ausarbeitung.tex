\documentclass[a4paper,nochapterprefix,ngerman,12pt]{scrreprt} 

\usepackage[german]{babel}
\usepackage[utf8]{inputenc}
\usepackage[usenames,dvipsnames]{xcolor}
\usepackage{fancyhdr}
\usepackage{amsmath}
\usepackage{graphicx}
\usepackage[colorinlistoftodos]{todonotes}
\usepackage{listings}
\lstset{language=bash}
\usepackage{glossaries}
\usepackage{cite}
\usepackage{placeins}
\usepackage{fixltx2e}
\usepackage{fontenc}
\usepackage{multirow}
%\usepackage{scrpage2}
%\clearscrheadfoot
%\pagestyle{scrheadings}
\usepackage[
top    = 2cm,
bottom = 3.5cm,
left   = 3cm,
right  = 3cm]{geometry}
\setcounter{secnumdepth}{4}
\parindent0pt 
\usepackage{hyperref} 

\lstdefinestyle{customc}{
	belowcaptionskip=1\baselineskip,
	breaklines=true,
	xleftmargin=\parindent,
	language=bash,
	showstringspaces=false,
	basicstyle=\footnotesize\ttfamily,
	keywordstyle=\bfseries\color{green!40!black},
	commentstyle=\itshape\color{purple!40!black},
	identifierstyle=\color{blue},
	stringstyle=\color{orange},
}

\lstdefinestyle{customasm}{
	belowcaptionskip=1\baselineskip,
	frame=L,
	xleftmargin=\parindent,
	language=[x86masm]Assembler,
	basicstyle=\footnotesize\ttfamily,
	commentstyle=\itshape\color{purple!40!black},
}

\lstset{escapechar=@,style=customc}



\makeglossaries
\newglossaryentry{ip} {name=IP, description={Internet Protocol}}

%Hannah
\newglossaryentry{aws} {name=AWS, description={Amazon Web Services}}
\newglossaryentry{hp} {name=HP, description={Hewlett-Packard , L.P.}}
\newglossaryentry{ceo} {name=CEO, description={Chief Executive Officer}}
\newglossaryentry{ec2} {name=Amazon EC2, description={Amazon Elastic Compute Cloud}}
\newglossaryentry{s3} {name=Amazon S3, description={Amazon Simple Storage Service}}
\newglossaryentry{iam} {name=Amazon IAM, description={AWS Identity and Access Management}}
\newglossaryentry{ebs} {name=Amazon EBS, description={Amazon Elastic Block Store }}
\newglossaryentry{api} {name=Amazon API, description={Application programming interface}}
\newglossaryentry{rest} {name=REST, description={Representational State Transfer}}
\newglossaryentry{iaas} {name=IaaS, description={Infrastructure as a Service}}
\newglossaryentry{saas} {name=SaaS, description={Software as a Service}}
\newglossaryentry{paas} {name=PaaS, description={Platform as a Service}}
\newglossaryentry{cpu} {name=CPU, description={Central Processing Units}}
\newglossaryentry{os} {name=OS, description={Operating System}}


\renewcommand*\glspostdescription{\dotfill}




\title{Vergleich\\von Cloud-Stack Lösungen}

\author{Belinic Vennesa}

\date{\today}

\def\BibTeX{{\rm B\kern-.05em{\sc i\kern-.025em b}\kern-.08em
    T\kern-.1667em\lower.7ex\hbox{E}\kern-.125emX}}

\newcommand{\command}[1]{{\texttt{\\\color{blue} #1\\}}}
\newcommand{\error}[1]{{\texttt{\\\color{red} #1\\}}}
\newcommand{\comment}[1]{{\texttt{\\\color{OliveGreen} #1\\}}}

\newcommand{\citeof}[2]{{
		\par \begingroup \leftskip=1cm \noindent \textit 
		''#1'' \cite{#2} \\
		\par \endgroup
	}}

% UseCase
% \insertpicture{mik.png}{Some picture}{\cite{bk_key}}{itm:pic1}{0.5}
\newcommand{\insertpicture}[5]{{
		\begin{figure}[!htb]
			\centering\includegraphics[width=#5\textwidth]{#1}
			\caption[#2 #3]{#2}
			\label{#4}
		\end{figure}
		\FloatBarrier
	}}

\pagestyle{fancy}
\fancyhf{} % clear all header and footer fields
\fancyfoot[L]{© Belinić Vennesa}
\fancyfoot[R]{\thepage / \pageref{lastpage}}

\renewcommand{\headrulewidth}{0pt}
\renewcommand{\footrulewidth}{0pt}


\begin{document}

\maketitle
\pagenumbering{Roman}

\newpage
\tableofcontents

\newpage
\pagenumbering{arabic}
%\ohead{\headmark}
%\automark{section}
%\ifoot{© Belinić Vennesa}
%\ofoot{\pagemark/n}

\chapter{Einfürhung} \thispagestyle{fancy}
Grundsätzlich handelt es sich bei allen Produkten um Cloud-Plattformen.

\chapter{Apache CloudStack} \thispagestyle{fancy}
\section{Historische Entwicklung \cite{apachehistory,apacheusers}}
Apache Cloudstack begann als Projekt des Start-Up Unternehmens VMOps und wurde im Jahr 2008 bekannt. Später änderte sich der Name des Unternehmens in Cloud.com und im Mai 2010 wurde der Großteil der Sources unter der GNU General Public License version 3 (GPLv3) veröffentlicht.\\\\
Im Juli 2011 wurde das Unternehmen Cloud.com von Citrix aufgekauft. Dieses veröffentlichte den Rest des Codes auch unter der GPLv3 im August 2011 und machte einen Release CloudStack 3.0 Anfang des Jahres 2012.\\\\
Im April 2012 wurde CloudStack neu veröffentlich unter der Apache Software License 2.0 (ASLv2) und an Apache Incubator (Apache Projekte die von externen Unternehmen finanziert werden, wo versucht wird eine Community aufzubauen) abgegeben. Nachdem eine Community und eine Infrastruktur aufgebaut war, wurde der erste groß stable Release gemacht am 6.November 2012.\\\\
Apache CloudStack hat das Incubator Programm erfolgreich ''absolviert''.\\\\\\
\textbf{Einige der Unternehmen die dieses Produkt benutzen:\\\\}
\begin{minipage}{.5\textwidth}
	\begin{itemize}
		\item DATACENTER Services
		\item UPCnet
		\item Apple
		\item Dell
		\item Disney
		\item EnterpriseDB
		\item Fujitsu FIP Corporation
		\item Logicworks
		\item Microland Ltd
		\item Nokia
	\end{itemize}
\end{minipage}
\begin{minipage}{.5\textwidth}
	\begin{itemize}
		\item OpenERP
		\item Orange
		\item SAP
		\item ScienceLogic, Inc.
		\item Shopzilla
		\item TomTom
		\item UniSystems
		\item Vision Solutions, Inc.
		\item VMTurbo
		\item Zynga
	\end{itemize}
\end{minipage}

\section{Lizenz}
\section{Features}
(welche as-a-Service Varianten werden unterstützt)
\section{Voraussetzungen}
(welche Virtualisierungs-Lösungen werden unterstützt/benötigt)
\section{Dokumentation}
Umfang und Qualität der

\chapter{Eucalyptus Systems} \thispagestyle{fancy}
Eucalyptus steht für Elastic Utility Computing Architecture for Linking Your Programs To Useful Systems. \cite{EucalyptusSlideShare}\\
Eucalyptus Systems wurde am 12.September 12 2014 von \gls{hp} gekauft. \cite{eucHP2}
HP hat Marten Mickos als \gls{ceo} eingesetzt. \cite{eucHP1}
\\
Eucalyptus Systems gibt Firmen die Möglichkeit, über eine open-source Software \gls{aws}-compartible private und hybrid clouds einzurichten. \\
Es werden die gängigen \gls{aws} \gls{api}s unterstützt, zum Beispiel \gls{ec2}, \gls{s3},\gls{iam} und \gls{ebs}
\cite{eucHP2}
\section{Historische Entwicklung}
Die erste Version wurde etwa 2008 von der University of California, Santa Barbara, (UCSB) entwickelt. \\
Die damaligen Ziele waren jedoch nicht, \gls{ec2} zu ersetzen sonder vielmehr einen Mehrwert hinzuzufügen.
Das Interface wurde an das von Amazon WSDL angepasst, und damals war die Unterstützung von \gls{s3} noch nicht realisiert, allerdings bereits angedacht. Jedoch hat Eucalyptus damals schon seinen eigenen Cloud Admin definiert. Die erste Realese wurde am 28. Mai 2008 veröffentlicht. \\
Version 1.1 am 1. Juli 2008 hatte Bug fixed, die implementierung von \gls{rest} interfaces sowie einen Source code release. Mit erstem Jänner 2009 war auch die Unterstützung von \gls{ebs} geplant. 
\cite{EucalyptusSlideShare} \\
2010 war Eucalyptus dann von Eucalyptus Inc. supportet. \cite{Eucalyptus2010}
\\ \\
Seit September 2014 als Teil von \gls{hp}, ist nun die Version 4 von Eucalyptus erhältlich. "Eucalytus 4.0 implementiert eine Cloud nach dem Servicemodell \gls{iaas}", \cite{EucalyptusV4}\\
Im Jahre 2012 wurde von Eucalyptus und Amazon eine technologische Partnerschaft angekündigt, in welcher sie erklären, dass auch in Zukunft von beiden Seiten eine Kompatibilität gewährleistet werden soll.\cite{EucalyptusAWSPart} \\
Daher stehen sowohl \gls{hp} als auch Amazon hinter Eucalyptus und so sind zwei starke Firmen vorhanden.\\
\\
Zu den von Eucalyptus angeführten Kunden zaehlen unter anderem:
\begin{itemize}
\item NASA
\item National Center for Atmospheric Research 
\item Nokia
\item Puma
\item Wirtschaftsuniveristät Wien (WU)
\item University of Oxford
\end{itemize}
\cite{EucalyptusCustomers}
\section{Lizenz}
\subsection{Ubuntu Enterprise Cloud (UEC)}
Die Ubuntu Enterprise Cloud (UEC) ist bei der Ubuntu Server Edition dabei. Canonical übernimmt den technischen support für UEC. \cite{EucalyptusBegGuide}
\subsection{Eucalyptus Enterprise Edition (Eucalyptus EE)}
Mit 16. Juni 2010 wurde von  Eucalyptus Systems, Inc. Eucalyptus Enterprise Edition (EE) 2.0 ein update für das damals schon vorhandene EE veröffentlicht. Es unterstützt Windows virtual machines, dadurch kann ein User nun auch Windows systeme verwenden. \cite{EucalyptusEELaunch}\\ \\
Es sind 3 Modelle vorhanden: \\
\begin{minipage}[t]{.2\textwidth}
\textbf{Community} \\
\textit{gratis} \\
Community Support\\
Security updates
\end{minipage}
\begin{minipage}[t]{.05\textwidth}
\hspace{1.0\textwidth}
\end{minipage}
\begin{minipage}[t]{.3\textwidth}
\textbf{Standard} \\
\textit{199\$ pro server / Monat} \\
Unlimitierter Web support \\
Sicherheitswarnungen \\
Advanced modules \\
Support an Werktagen
\end{minipage}
\begin{minipage}[t]{.05\textwidth}
\hspace{1.0\textwidth}
\end{minipage}
\begin{minipage}[t]{.3\textwidth}
\textbf{Premium} \\
\textit{299\$ pro server / Monat} \\
24/7 Support \\
Schnelle Bearbeitung
\end{minipage}
\cite{EucalyptusPric}
\section{Features}
Eucalyptus ist besonders auf \gls{iaas} spezialisiert.
\section{Voraussetzungen}
Pro Server gilt:
\begin{itemize}
\item \gls{cpu}: Mindestens zwei, 2GHz Kerne
\item \gls{os}:  CentOS 6 und RHEL 6. nur 64-bit Architekturen werden unterstützt
\item Die internen clocks müssen synchronisiert sein.
\item Jede Maschine benötigt einen root zugriff über SSH
\item Mindestens 30GB speicher (zwischen 100-250 empfohlen)
\item Mindestens 4GB RAM
\item Mindestens 1Gb Ethernet Netzwerk Anschluss
\end{itemize}
Andere Vorraussetzungen für spezielle Features existieren. \cite{EucalyptusRequ}
\section{Dokumentation}
Die Dokumentation kann online (\cite{EucalyptusDoc}) eingesehen werden.
Sie ist sehr umfangreich und auf englisch.


\chapter{Open Stack} \thispagestyle{fancy}
\section{Historische Entwicklung}
Im Jahre 1996 wurde das Unternehmen Cymitar Technology Group von Richard Yoo gegründet, welches der Ursprung von Rackspace war.
Ein unternehmen, dass sich hauptsächlich mit der Entwicklung von Websites beschäftigte.\\
Diese, zwischenzeitlich in die Tochtergesellschaft Mosso ausgelagerte Tätigkeit, entwickelte sich rasch zu einem Führenden Anbierter im Berreich Webhosting, da sie Vorreiter als Anbierter von gehostetem Webspace waren.\\
Nachdem Mosso zu einem bekannten Unternehmen in seiner Branche wurde, wurde es 2008 von Rackspace zurückgekauft und bildete somit die Basis für die Rackspace Cloud.\\
In Kooperation mit der NASA, Dell und Citrix Systems wurde im Anschluss das Open-Source-Projekt Open Stack iniziiert, welches bis heute bereits zehn Releases hervorbrachte.\\
\\
Folgende Unternehmen waren während der Entwickling involviert: \\

\begin{table}[h]
\begin{tabular}{cccc}
\textbf{Open  SUSE Gmbh} & \textbf{Canonical} & \textbf{Hewlett-Packard} &              \\
\textbf{AMD}             & \textbf{Intel}     & \textbf{Red Hat}         & \textbf{IBM}
\end{tabular}
\end{table}
\section{Lizenz}
Open Stack ist als freie Software unter der Apache-Lizenz veröffentlicht, womit sie frei verwendet, modifiziert und verteilt werden darf, sofern der Copyright Owner genannt wird und eine Kopie der lizenz beiliegt.
\section{Features}
(welche as-a-Service Varianten werden unterstützt)
\section{Voraussetzungen}
Das System läuft ausschließlich auf Linux Sytsemen und wird mit Ubuntu 14.04 empfohlen.\\
Auf den folgenden Distributionen ist Open Stack allerdings ebenfalls gepackaged und lauffähig:
\begin{itemize}
	\item Fedora 20
	\item CentOS/RHEL 7
	\item Open SUSE
	\item Debian
\end{itemize}
Genaue Hardwareanforderungen für die Verwendung des Systems findest man leider kaum, allerdings wird im eigenen Q&A-Berreich geschätzt, dass es rund vier bis sechs Gigerbyte sind.\\
Da das Sytsem auf Python basiert wird solch eine Version benötigt, Empfohlen wird hierbei 2.6 oder 22.7, wobei die Unterstützung von Jango ein Musskriterium ist.
Für die Minimalrealisierung müssen die Komponente Compute und der Identity Service installiert werden. Der Hersteller schätzt den Umfang dieser Installation auf vier bis sechs Gigerbyte, gibt allerdings keine Informationen über etaige andere Hardware anforderungen.\\
Open Stack steht allerdings auch als virtuelle Instanz zur Verfügung und bietet so jedem die Möglichkeit schnell einen Einblick zu bekommen.
\section{Dokumentation}
Open Stack geniest eine Umfangreiche Dokumentation über die API's seiner Komponenten und liefert darüber hinaus detailiierte, textuelle beschreibungen der Funktionsweise, sowie Code-Beispiele zu den Methoden.\\
Die folgenden Linux Umgebungen sind ofiziell dokumentiert:
\begin{itemize}
	\item Ubuntu 14.04
	\item Fedora 20
	\item CentOS/RHEL 7
\end{itemize}
Auf den Distributionen Open SUSE und Debian ist Open Stack zwar ebenfalls gepackaged und Lauffähig, allerdings ist man bei Hilfe auf Foren und Communities angewiesen.
Darüber hinaus stehen dem Entwickler ein Wiki und ein Forum zur Verfügung, um mehr wissen zu erlangen oder sich mit anderen Auszutauschen.\\
Weites werden Seminare und Schulungen angeboten um sich über das System weiter zu bilden.

\chapter{Fazit} \thispagestyle{fancy}

\bibliography{sources} \thispagestyle{fancy}
\bibliographystyle{alpha} \thispagestyle{fancy}

\listoffigures \thispagestyle{fancy}

\printglossary[style=tree,title={Glossar}]  \thispagestyle{fancy}

\label{lastpage}

\end{document}
